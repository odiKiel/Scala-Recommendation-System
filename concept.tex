%% LyX 2.0.0 created this file.  For more info, see http://www.lyx.org/.
%% Do not edit unless you really know what you are doing.
\documentclass[ngerman,english]{article}
\usepackage[T1]{fontenc}
\usepackage[latin9]{inputenc}
\usepackage{babel}
\usepackage{todonotes}
\bibliographystyle{gerplain}
\begin{document}

\title{A Contribution to Rating and Recommendation Systems: Concepts, Development
and Evaluation}

\maketitle

\section*{Concepts}

\subsection*{Taggen}
Deprecated!
Mithilfe eines part of speech tagger (POS Tagger) alle begriffe eines
Satzes taggen und ein minimales subset der W�rter eines Satzes gegen
die Begriffe des ZBW Thesaurus Matchen und den Deskriptoren zuordnen.
Somit kann man sp�ter den 20000 Synonymen die den 6000 Deskriptoren
zugeordnet sind die Informationen aus der Datenbank zuordnen.
Taggen: Triple store database mit thesaurus. All Woerter werden gegen den thesaurus gematched. Wenn treffer dann wird das prefered Label als tag verwendet.

\subsection*{Rating}
\begin{itemize}
\item Seiten Aufruf
\item aktive Aufenthaltsdauer
\item Scrollverhalten
\item benutzerdefinierte Tasks
\end{itemize}
Ein Rating ist ein Wert zwischen 0 und 10. Null bedeutet nicht vorhanden,
10 bedeutet max rating wird in der Regel nur durch benutzerdefinierte
tasks erreicht.

Jeder standard task hat einen standard Wert welcher beim Einbinden
des skripts abge�ndert werden kann.


\subsection*{Recommendation}


\subsubsection*{Collaborative recommendation}


\subsubsection{User-Based (Usually memory based)}

Finde Benutzer die �hnliche Bewertungen in der Vergangenheit abgegeben
haben. Anschlie�end berechne f�r jedes Produkt p welches der Benutzer
noch nicht gesehen hat einen Wert in Abh�ngigkeit der Bewertungen
der �hnlichen Benutzer.


\subsubsection{Item-Based (Usually model based )}

Finde Objekte die �hnliche Bewertungen haben wie ein noch nicht gesehenes
Produkt. Betrachte die Bewertungen des Benutzers f�r diese �hnlichen
Produkte berechne den Durchschnitt.


\subsubsection{Matrix factorization ( Usually model based )}

SVD Matrizen Zerlegung. Zerlege eine Bewertungstabelle so, dass M=
U{*}Sigma{*}V\textasciicircum{}T Betrachte nur die obersten Spalten
von U und V und approximiere mit diesen die Benutzer sowie die Items.


\subsubsection{Probabilistic Approach ( Usually model based)}


\subsubsection{Slope One ( Usually model based )}

Problem: Viele Daten viele Benutzer und vorraussichlich viele neue
Informationen/Benutzer pro Tag. Daher ist ein rein model basierter
Ansatz nicht aktuell genug. Ein rein memory basierter Ansatz ist mit
einer normalen Serverauslastung nicht m�glich daher muss ein hybrider
Ansanz gew�hlt werden.


\subsection*{Allgemeiner Aufbau}
Programmiersprache: Scala
Framework: Finagle
Service Architektur



\subsubsection*{Fragen Taggen }

Deprecated
Plugin mit python mithilfe des Natural language toolkits. Eine Frage
bekommt eine many to many connection zu einem Datenbankeintrag mit
einem Deskriptor.

Jedem prefered label sind n Synonyme zugeordnet. 
Cronjob, der alle 30min die Datenbank nach neuen eintraegen ueberprueft. Wenn ein neuer Eintrag vorhanden ist matcht er alle Woerter des Eintrags gegen die Woerter des thesaurus. Die prefered label der gematchten Woerter werden als tag verwendet.


\subsubsection*{Javascript Clientseitig}

Javascript script in website einbinden angaben rating:
\begin{itemize}
\item id des objekts
\item css selector f�r rating (Beispiel button wird geklickt starkes Rating
von 0-10)
\item standard Ratings k�nnen aktiviert/deaktiviert werden
\item standard Rating Werte k�nnen ver�ndert werden
\end{itemize}
Standard Ratings sind: 
\begin{itemize}
\item Seiten Aufruf
\item aktive Aufenthaltsdauer
\item scrollverhalten
\end{itemize}
beim Verlassen der Seite werden die Daten an das recommendation system
�bertragen

Angaben Ausstreuen der Informationen:
\begin{itemize}
\item Angabe eines css selectors mit divboxen in denen die Information eingebettet
werden soll.
\selectlanguage{ngerman}%
\item Angabe\foreignlanguage{english}{ des }thematischen\foreignlanguage{english}{
Bezuges damit thematisch passende Informationen eingeblendet werden
k�nnen.}
\end{itemize}

\subsection*{Ideen}

\subsubsection*{String match}

http://en.wikipedia.org/wiki/Levenshtein\_distance bei distance groesser
4 abbrechen

Bitap algorithm
\subsubsection{Summaries}<++>
\bold{SCENE : A Scalable Two-Stage Personalized News Recommendation System}<++>
Difference betwenn news recommendation and qa-recommendation: for news recommendation the publishing date is very important therefore there are many new articles for every user. With this in mind it is important to know the subject that the user is interested in. However at qa-recommendation it is important to know similiar users but the overall interest of the user is not that important because the subject of the recommended item is based upon the theme of the currently visited website.
Quote: 'What differentiates our work from prior methods is that we model personalized news recommendation as a budgeted maximum coverage problem, i.e., the selection of one news item will influence the selection of the following news items.'
Two-stage procedure at first the news articles are clustered
\bold{Personalized News Recommendation Based on Click Behavior}<++>
Idee: Fragen die haeufig betrachtet werden erhalten ein hoeheres ranking.
\end{document}
