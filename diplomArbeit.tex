%%%%%%%%%%%%%%%%%%%%%%%%%%%%%%%%%%%%%%%%%%%%%%%%%%%%%%%%%%%%%%%%%%
%%                                                              %%
%% Template f�r Studien-, Diplom-, Bachelor- und Masterarbeiten %%
%%                 am Lehrstuhl ISE der CAU Kiel                %%
%%                                                              %%
%%                              17.10.2006                      %%
%%                                                              %%
%%%%%%%%%%%%%%%%%%%%%%%%%%%%%%%%%%%%%%%%%%%%%%%%%%%%%%%%%%%%%%%%%%
 
%% Stellen, an denen noch Daten eingetragen werden m�ssen,
%% sind mit ** gekennzeichnet.

%% Gr��e: A4, doppelseitig 
\documentclass[twoside,a4paper,BCOR1.0cm]{scrbook}

\usepackage{ifpdf}

\usepackage{ae,aecompl}
\usepackage[latin1]{inputenc} %% unter Linux/Unix "ansinew" evtl. durch "latin1" ersetzen
%%\usepackage{ngerman}
%%\usepackage{url}

%% Falls PicTeX-Grafiken (z.B. aus XFig) eingebunden werden sollen
%%\usepackage{pictexwd}

%% einige mathematische Symbole (falls ben�tigt)
%%\usepackage{stmaryrd}
%%\usepackage[intlimits]{amsmath}
%%\usepackage{amssymb}

%% f�r listings-Umgebungen und Algorithmen
%%\usepackage{algorithmic}
%%\usepackage{algorithm}
%%\usepackage{listings}

%%\renewcommand{\listalgorithmname}{Listingsverzeichnis} 

%% Literatur-Verzeichnis
\usepackage{natbib}
\bibpunct{[}{]}{;}{a}{,}{,}

\ifpdf
  % f�r Grafiken
  \usepackage[pdftex]{graphicx}
  
  % Verweise in PDF
  \usepackage[pdftex,plainpages=false]{hyperref}
  \pdfcompresslevel=9  

  % Metadaten des Dokuments
  \hypersetup{%
    a4paper,
    pdftitle = {** Hier den Titel eintragen **},
    pdfsubject = {** Hier eine Beschreibung eintragen **},
    pdfkeywords = {** Hier Schlagw�rter eintragen **},
    pdfauthor = {** Hier den Namen des Autors eintragen **},
    % pdfpagemode = None, UseThumbs, UseOutlines, FullScreen,
    % pdfstartpage = ,
    % pdfstartview = 
  }
\else
  \usepackage[plainpages=false]{hyperref}
  \usepackage{graphicx}
\fi    

\setlength{\parindent}{0cm}

\begin{document}
\ifpdf
  \DeclareGraphicsExtensions{.jpg, .pdf, .mps, .png}
\else
  \DeclareGraphicsExtensions{.eps}
\fi

\renewcommand{\textfraction}{0.1}

%%%%%%%%%%%%%%%%%%%%%%%%%%%%%%%%%%%%%%%%%%%%%%%%%%%%%%%%%%%%%%%%%%
%%                                                              %%
%%                          Titelseite                          %%
%%                                                              %%
%%%%%%%%%%%%%%%%%%%%%%%%%%%%%%%%%%%%%%%%%%%%%%%%%%%%%%%%%%%%%%%%%%

\pagestyle{empty}

\begin{center}
{\huge \it **[Studien|Diplom|Bachelor|Master]arbeit**}

\vspace{2cm}

{\Large \bf ** Titel der Arbeit **}

\vspace{2.25cm}

%%\includegraphics[height=4cm]{CAU-Siegel}

\vspace{2.25cm}

{\large 
Christian-Albrechts-Universit�t zu Kiel \\
Institut f�r Informatik  \\
Lehrstuhl Technologie der Informationssysteme
}

\end{center}

\vspace{2cm}

\begin{tabular}{ll}
angefertigt von:             & {\bf ** eigener Name **} \\
betreuender Hochschullehrer: & ** Name des betreuenden Hochschullehrers ** \\%
%                              z.B. Prof. Dr. rer. nat. habil. Bernhard Thalheim 
%                              oder Prof. Dr. rer. nat. habil. Hans-Joachim Klein
Betreuer:                    & ** Name des Betreuers ** 
\end{tabular}

\vspace{1cm}

\begin{center}
Kiel, ** Datum der Abgabe **
\end{center}


\cleardoublepage

%%%%%%%%%%%%%%%%%%%%%%%%%%%%%%%%%%%%%%%%%%%%%%%%%%%%%%%%%%%%%%%%%%
%%                                                              %%
%%                      Aufgabenstellung                        %%
%%                                                              %%
%%%%%%%%%%%%%%%%%%%%%%%%%%%%%%%%%%%%%%%%%%%%%%%%%%%%%%%%%%%%%%%%%%

\pagestyle{plain}
\chapter*{Aufgabe}

\begin{tabular}{ll}
{\bf Name, Vorname: }               & ** Name, Vorname **                            \\
{\bf Immatrikulations-Nr: }         & ** Immatrikulations-Nr **                      \\
{\bf Studiengang: }                 & ** Studiengang **                              \\
                                    &                                                \\
{\bf betreuender Hochschullehrer: } & ** Name des Hochschullehrers **   \\
{\bf Betreuer: }                    & ** Name des Betreuers **                       \\
{\bf Institut: }                    & Institut f�r Informatik  \\
{\bf Arbeitsgruppe: }               & Technologie der Informationssysteme            \\
                                    &                                                \\
{\bf Beginn am: }                   & ** Datum des Beginns **                        \\
{\bf Einzureichen bis: }            & ** Abgabetermin **                           
\end{tabular}

\vspace{1cm}

{\bf Aufgabenstellung:}

\vspace{0.5cm}

** Text der Aufgabenstellung **

\pagenumbering{roman}
\setcounter{page}{1}
\cleardoublepage

%%%%%%%%%%%%%%%%%%%%%%%%%%%%%%%%%%%%%%%%%%%%%%%%%%%%%%%%%%%%%%%%%%
%%                                                              %%
%%                Selbstst�ndigkeitserkl�rung                   %%
%%                                                              %%
%%%%%%%%%%%%%%%%%%%%%%%%%%%%%%%%%%%%%%%%%%%%%%%%%%%%%%%%%%%%%%%%%%

\chapter*{Selbstst�ndigkeitserkl�rung}

\vspace{1.5cm}

Ich erkl�re hiermit, dass ich die vorliegende Arbeit selbstst�ndig und nur unter Verwendung der angegebenen Literatur und Hilfsmittel angefertigt habe.

\vspace{2cm}
............................................................... \\
** eigener Name **

\thispagestyle{plain}
\cleardoublepage

%%%%%%%%%%%%%%%%%%%%%%%%%%%%%%%%%%%%%%%%%%%%%%%%%%%%%%%%%%%%%%%%%%
%%                                                              %%
%%                     Inhaltsverzeichnis                       %%
%%                                                              %%
%%%%%%%%%%%%%%%%%%%%%%%%%%%%%%%%%%%%%%%%%%%%%%%%%%%%%%%%%%%%%%%%%%

\tableofcontents

\cleardoublepage

%%%%%%%%%%%%%%%%%%%%%%%%%%%%%%%%%%%%%%%%%%%%%%%%%%%%%%%%%%%%%%%%%%
%%                                                              %%
%%                     Abbildungsverzeichnis                    %%
%%                                                              %%
%%%%%%%%%%%%%%%%%%%%%%%%%%%%%%%%%%%%%%%%%%%%%%%%%%%%%%%%%%%%%%%%%%

\listoffigures

\cleardoublepage

%%%%%%%%%%%%%%%%%%%%%%%%%%%%%%%%%%%%%%%%%%%%%%%%%%%%%%%%%%%%%%%%%%
%%                                                              %%
%%                     Tabellenverzeichnis                      %%
%%                                                              %%
%%%%%%%%%%%%%%%%%%%%%%%%%%%%%%%%%%%%%%%%%%%%%%%%%%%%%%%%%%%%%%%%%%

\listoftables

\cleardoublepage

%%%%%%%%%%%%%%%%%%%%%%%%%%%%%%%%%%%%%%%%%%%%%%%%%%%%%%%%%%%%%%%%%%
%%                                                              %%
%%                   Algorithmenverzeichnis                     %%
%%                                                              %%
%%%%%%%%%%%%%%%%%%%%%%%%%%%%%%%%%%%%%%%%%%%%%%%%%%%%%%%%%%%%%%%%%%

%%\listofalgorithms

%%\cleardoublepage

%%%%%%%%%%%%%%%%%%%%%%%%%%%%%%%%%%%%%%%%%%%%%%%%%%%%%%%%%%%%%%%%%%
%%                                                              %%
%%              Liste der verwendeten Abk�rzungen               %%
%%                                                              %%
%%%%%%%%%%%%%%%%%%%%%%%%%%%%%%%%%%%%%%%%%%%%%%%%%%%%%%%%%%%%%%%%%%

\chapter*{Abk�rzungen}


\cleardoublepage

%%%%%%%%%%%%%%%%%%%%%%%%%%%%%%%%%%%%%%%%%%%%%%%%%%%%%%%%%%%%%%%%%%
%%                                                              %%
%%                       Text der Arbeit                        %%
%%                                                              %%
%%%%%%%%%%%%%%%%%%%%%%%%%%%%%%%%%%%%%%%%%%%%%%%%%%%%%%%%%%%%%%%%%%

\pagestyle{headings}
\pagenumbering{arabic}
\setcounter{page}{1}

\chapter{Einf�hrung}

** Jetzt geht's los! **

\chapter{Zweites Kapitel}
\section{Erster Unterabschnitt}
\section{Zweiter Unterabschnitt}
\section{Dritter Unterabschnitt}

\chapter{Concepts}
%%todo check number of labels from the stw thesaurus together with a source
\section{Tagging}
The purpose of tagging is to find the right keywords for a text. For the context of economics the \emph{STW Thesaurus for Economics} includes 16000 labels which can be used to match words from a text and provide therefore a good tagging base. However the \emph{STW Thesaurus for Economics} only includes the basic form of the words therefore a normal test for equal is not sufficient enough. For example a word that is used in it's plural form in a text would return false on a test for equal and would therefore not be a candidate for a tag for the text. This creates the need for an algorithm that indicates the distance between two words. A practical choice for such a task is the \emph{levenshtein distance algorithm}.
\subsection{Levenshtein Distance Algorithm}
The \emph{levenshtein distance algorithm} calculates the minimum numbers of substitutions, insertions and deletions that are needed to create one word into another. Therefore the algorithm returns zero if the words are equal and adds one to the result if it has to perform a substitution an insertion or a deletion of a letter.

todo add pseudocode algorithm
However the direct implementation of the \emph{levenshtein distance algorithm} has a complexity of O(something). An optimized version of the \emph{levenshtein distance algorithm} that uses a deterministic automaton was introduced in \emph{Fast string correction with Levenshtein automata by Schulz and Mihov} and calculates a deterministic finite automata with the size of the maximum allowed corrections. It uses letters as inputs and indicates in it's state the current letter count and the number of necessary corrections needed to reach the current state. A final state is a state that has the letter length of the original word and the number of corrections is smaller than the allowed corrections.

Lemma 6 For any �xed number n, given two words W
and V of length w and v respectively, it is decidable in
time O(max(w, v)) if the Levenshtein distance between
W and V is � n.

The current system implementation calculates for every word from the text an automata and reads all words from the \emph{STW Thesaurus for Economics}. If a word from the \emph{STW Thesaurus for Economics} reaches a final state 

\appendix

\chapter{Erster Anhang}

\chapter{Zweiter Anhang}

%%%%%%%%%%%%%%%%%%%%%%%%%%%%%%%%%%%%%%%%%%%%%%%%%%%%%%%%%%%%%%%%%%
%%                                                              %%
%%                     Literaturverzeichnis                     %%
%%                                                              %%
%%%%%%%%%%%%%%%%%%%%%%%%%%%%%%%%%%%%%%%%%%%%%%%%%%%%%%%%%%%%%%%%%%

\bibliographystyle{alpha}
\bibliography{mybib}       %% statt mybib Name der eigenen .bib-Datei einsetzen 

\end{document}
