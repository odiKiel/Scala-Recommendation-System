\documentclass{beamer}
\usepackage{lmodern}% http://ctan.org/pkg/lm
\usepackage{amsmath}
\begin{document}
\title{A Contribution to Rating and Recommendation Systems: Concepts, Development and Evaluation}   
\author{Oliver Diestel} 
\date{\today} 

\frame{\titlepage} 

%\frame{\frametitle{Inhaltsverzeichnis}\tableofcontents} 


\section{Overview} 
\frame{\frametitle{Overview} 
%Picture of Website / QA with ratings information
Website and Rating
page 1 general overview what do we display on which page?
image qa / website we display questions on website
}

\frame{\frametitle{Rating}
\begin{itemize}
\item Click 
\item active time on page 
\end{itemize}

If a user stays shorter or longer on a page than the average user -  
it is a high or low rating.\\
We rate QA pages and webpages, however we only recommend qa pages.

}

\frame{\frametitle{Tagging}
\begin{itemize}
\item Standard-Thesaurus Wirtschaft, tripple store
\item Stemming: Compute the root of a word.
\item Computing Levenshtein distance: Calculate the distance between two words.
\end{itemize}
\begin{block}{Principle}
Every question will be tagged with a prefered label from the ZBW Thesaurus.
For every word in the question find the root of the word and match it against the thesaurus. If we find a matching prefered label, Done. If we find a matching synonym take the preferd label as tag, Done.
\end{block}

}

\section{Recommendation}
\frame{\frametitle{Recommendation}
Three part recommendation. Pipeline architecture
\begin{itemize}
\item Calculate similarity between items
\item Add similar ratings to a user/rating vector.
\item Calculate singular value decomposition
\item Find similar users 
\end{itemize}
}
\frame{\frametitle{Input/Output}
Input: $\langle$user\_id, item\_id, rating$\rangle$, [List of tags], number of recommendations\\
Output: Sorted list of Items with avg. rating that correspond with the pref. input label
\begin{itemize}
\item Item: id, name, tag
\item User: id, name
\end{itemize}
}


\frame{\frametitle{Item-Based Algorithm}
Find items that have similar ratings.
Calculate the similarity values for every item.
\begin{block}{Cosinus Similarity}
\begin{math}
  sim(\overrightarrow{a}, \overrightarrow{b}) = {\overrightarrow{a} * \overrightarrow{b} \over |\overrightarrow{a}|*|\overrightarrow{b}|}
\end{math}
\end{block}

Take the differences of the average rating behaviour of the user into account.

\begin{block}{Advanced Cosinus Similarity}
\begin{math}
  sim(a, b) = {\sum_{u \in U}(r_{u, a} - \overline{r_u})(r_{u,b} - \overline{r_u}) \over \sqrt{\sum_{u \in U}(r_{u,a} - \overline{r_u})^2} \sqrt{\sum_{u \in U}(r_{u,b} - \overline{r_u})^2 }}
\end{math}
\end{block}

Calculate predictions for similar items.

User u, Product p
\begin{math}
  pred(u, p) = {\sum_{i \in ratedItems(u)} sim(i, p) * r_{u, i} \over \sum_{i \in ratedItems(u)} sim(i, p)}
\end{math}

Fill recommendation vector of each user with similar item ratings 
}

\frame{\frametitle{Singular Value Decomposition}
Create a SVD with the matrix 
\begin{math}
  M = U * \Sigma * V^t
 \end{math}
 %explanations for u sigma and v
\begin{itemize}
  \item U dimension m x m, orthogonal matrix, spans the column space of matrix m
  \item $\Sigma$ dimension m x n, diagonal matrix having only r nonzero entries
  \item V dimension n x n, orthogonal matrix, spans the row space of matrix m
\end{itemize}
\begin{block}{$\Sigma$ Properties}
  Diagonal entries of $\Sigma$ have the property, that $\sigma > 0$ and $\sigma_i \geq \sigma_{i+1}$
\end{block}
 
 Calculate cosinus similarity between users
}

\frame{\frametitle{Find Questions}
Find questions that are hightly rated by similar users that match the current topic
}

\frame{\frametitle{Display Questions}
Add a tag and  a div-placeholder on every webpage. Display top questions for the current user on the webpage.

}

% \section{Abschnitt Nr. 2} 
% \subsection{Listen I}
% \frame{\frametitle{Aufz\"ahlung}
% \begin{itemize}
% \item Einf\"uhrungskurs in \LaTeX  
% \item Kurs 2  
% \item Seminararbeiten und Pr\"asentationen mit \LaTeX 
% \item Die Beamerclass 
% \end{itemize} 
% }

% \frame{\frametitle{Aufz\"ahlung mit Pausen}
% \begin{itemize}
% \item  Einf\"uhrungskurs in \LaTeX \pause 
% \item  Kurs 2 \pause 
% \item  Seminararbeiten und Pr\"asentationen mit \LaTeX \pause 
% \item  Die Beamerclass
% \end{itemize} 
% }

% \subsection{Listen II}
% \frame{\frametitle{Numerierte Liste}
% \begin{enumerate}
% \item  Einf\"uhrungskurs in \LaTeX 
% \item  Kurs 2
% \item  Seminararbeiten und Pr\"asentationen mit \LaTeX 
% \item  Die Beamerclass
% \end{enumerate}
% }
% \frame{\frametitle{Numerierte Liste mit Pausen}
% \begin{enumerate}
% \item  Einf\"uhrungskurs in \LaTeX \pause 
% \item  Kurs 2 \pause 
% \item  Seminararbeiten und Pr\"asentationen mit \LaTeX \pause 
% \item  Die Beamerclass
% \end{enumerate}
% }

% \section{Abschnitt Nr.3} 
% \subsection{Tabellen}
% \frame{\frametitle{Tabellen}
% \begin{tabular}{|c|c|c|}
% \hline
% \textbf{Zeitpunkt} & \textbf{Kursleiter} & \textbf{Titel} \\
% \hline
% WS 04/05 & Sascha Frank &  Erste Schritte mit \LaTeX  \\
% \hline
% SS 05 & Sascha Frank & \LaTeX \ Kursreihe \\
% \hline
% \end{tabular}}


% \frame{\frametitle{Tabellen mit Pause}
% \begin{tabular}{c c c}
% A & B & C \\ 
% \pause 
% 1 & 2 & 3 \\  
% \pause 
% A & B & C \\ 
% \end{tabular} }


% \section{Abschnitt Nr. 4}
% \subsection{Bl\"ocke}
% \frame{\frametitle{Bl\"ocke}

% \begin{block}{Blocktitel}
% Blocktext 
% \end{block}

% \begin{exampleblock}{Blocktitel}
% Blocktext 
% \end{exampleblock}


% \begin{alertblock}{Blocktitel}
% Blocktext 
% \end{alertblock}
% }
\end{document}

